\chapter{Natural numbers}

\section{Definition}

\subsection{Data structure}
\begin{coq}{}{}
Inductive nat : Type := O | S (n : nat)
\end{coq}

\subsection{Notations for the constructors}

\begin{tabular}{llll}
Notation  & Term      & Type & Meaning \\\hline
\C{0}     & \C{O}     & \C{nat} & Zero \\\hline
\C{1}, \C{2},\ldots   & \C{(S O)},\C{(S (S O))},\ldots & \C{nat} & $2, 3,\ldots$\\\hline
\C{n.+1}              & \C{(S n)} & \C{nat} & $n+1$ \\\hline
\C{n.+2}              & \C{(S (S n))} & \C{nat} & $n+2$ \\\hline
\C{n.+3}              & \C{(S (S (S n))))} & \C{nat} & $n+3$ \\\hline
\C{n.+4}              & \C{(S (S (S (S n)))))} & \C{nat} & $n+4$ \\\hline
\end{tabular}

\subsection{Basic associated commands}

\paragraph{Case analysis}
On a natural number \C{n : nat} present in the context, the tactic:
\begin{coq}{}{}
  case: n => [|m].
\end{coq}
opens two subgoals: in the first one \C{n} is substituted
by \C{O}, in the second one it is substituted by \C{m.+1}.


\paragraph{Recurrence}
On a natural number \C{n : nat} present in the context, the tactic:
\begin{coq}{}{}
  elim: n => [|m ihm].
\end{coq}
opens two subgoals: in the first one \C{n} is substituted
by \C{O}, in the second one it is substituted by \C{m.+1}. The second
subgoal features an induction hypothesis, for \C{m}, named \C{ihm}.

\paragraph{See also}
Section~\ref{sec:nat-useful-tac}.

\section{Comparisons}

\begin{tabular}{llll}
Notation  & Term      & Type & Meaning \\\hline
\C{n = m} & \C{(eq n m)} & \C{Prop} & Rewritable equality \\
\C{n <> m} & \C{(not (eq n m))} & \C{Prop} & Disequality\\\hline
\C{n == m}  & \C{(eqn n m)}  & \C{bool} & Boolean equality \\
\C{n != m}  & \C{(negb (eqn n m))}  & \C{bool} & Boolean disequality \\\hline
\C{n <= m}  & \C{(leq n m)}  & \C{bool} & $n \leq m$ \\\hline
\C{n < m}  & \C{(leq n m)}   & \C{bool} & $n < m$ \\\hline
\C{n >= m}  & \C{(geq n m)}  & \C{bool} & $n \geq m$ \\\hline
\C{n > m}  & \C{(geq n m)}   & \C{bool} & $n > m$ \\
\end{tabular}

\paragraph{See also}
Section~\ref{nat:more-rel}.

\section{Basic arithmetics}

\begin{tabular}{llll}
Notation  & Term      & Type & Meaning \\\hline
\C{n.-1}  & \C{(predn n)} & \C{nat} & $n - 1$ \\
\C{n.-2}  & \C{(predn (predn n))} & \C{nat} & $n - 2$ \\
\C{(n + m)} & \C{(addn n m)} & \C{nat} & $n + m$ \\
\C{(n * m)} & \C{(muln n m)} & \C{nat} & $n * m$ \\
\C{(n - m)} & \C{(subn n m)} & \C{nat} & $n - m$, \quad $0$ if $m \leq n$\\
\end{tabular}


\section{Declared canonical structures}
\begin{itemize}
\item Equality type (\C{eqtype});
\item Countable type (\C{counttype});
\item Choice type (\C{choicetype});
\item Bigops for addition, multiplication, gcd, lcm, max.
\end{itemize}

\section{More operations}

\begin{tabular}{llll}
Notation  & Term      & Type & Meaning \\\hline
$\_$      & \C{minn m n} & \C{nat} & Minimum of $n$ and $m$  \\\hline
$\_$      & \C{maxn m n} & \C{nat} & Maximum of $n$ and $m$  \\\hline
\C{m ^ n} & \C{(expn m n)} & \C{nat} & $m^n$\\\hline
\C{n\`!}   & \C{factorial n)} & \C{nat} & $n!$\\\hline
\C{n.*2}  & \C{(double n)} & \C{nat} & $2n$ \\\hline
\C{n./2}  & \C{(half n)} & \C{nat} & $\floor{\frac{n}{2}}$ \\\hline
$\_$ & \C{(uphalf n)} & \C{nat} &  $\floor{\frac{n + 1}{2}}$ \\\hline
$\_$ & \C{(odd n)} & \C{bool} & Tests the oddity of $n$ \\\hline
\C{m \%/ d} & \C{(divn m d)} & \C{nat} &  quotient of the Euclidean
                                         division of $m$ by
                                         $d$\\
\C{m \%\% d} & \C{(modn m d)} & \C{nat} & remainder of the Euclidean division of $m$ by $d$\\
$\_$      & \C{(edivn m n)} & \C{nat * nat} & Pair the quotient and remainder\\
                     &&& of the Euclidean division of $m$ by $n$\\\hline
$\_$      & \C{(gcdn m n)} & \C{nat} & GCD of $m$ and $n$\\
$\_$      & \C{(egcdn m n)} & \C{nat * nat} & Bézout coefficients of $m$ and $n$\\
$\_$      & \C{(lcmn m n)} & \C{nat} & LCM of $m$ and $n$\\\hline

\end{tabular}
\section{More relations}\label{nat:more-rel}

\begin{tabular}{llll}
\C{m = n \%[mod d]} & \C{(m \%\% d = n \%\% d)}  & \C{Prop} & $m$ equals $n$ modulo $d$\\
\C{m == n \%[mod d]}& \C{(m \%\% d == n \%\% d)}  & \C{bool} & $m$
                                                               equals
                                                               $n$
                                                               modulo
                                                               $d$ \\
&&& (boolean version) \\
\C{m <> n \%[mod d]} & \C{(m \%\% d <> n \%\% d)}  & \C{Prop} & $m$ differs
           from $n$ modulo $d$\\
\C{m <> n \%[mod d]} & \C{(m \%\% d <> n \%\% d)}  & \C{Prop} & $m$ differs
          \%from $n$ modulo $d$ \\
&&& (boolean version)\\\hline
\C{m <= n <= p} & \C{((m <= n) && (n <= p))} & \C{bool} & $m \leq n \leq p$\\
\C{m <= n < p} & \C{((m <= n) && (n < p))} & \C{bool} & $m \leq n < p$\\
\C{m < n <= p} & \C{((m < n) && (n <= p))} & \C{bool} & $m < n \leq p$\\
\C{m < n < p} & \C{((m < n) && (n < p))} & \C{bool} & $m < n < p$\\
\C{ m <= n ?= iff Cond} & \C{(leqif m n Cond)} & \C{Type} & $m \leq n$
                                                            and $m =
                                                            n$ iff
                                                            Cond (pair
                                                            of rewrite
                                                            rules)
  \\\hline
\C{d \%| m} & \C{(divn d m)}  & \C{bool} & $d$ divides $m$ \\\hline
\end{tabular}

\section{Useful tactic idioms}\label{sec:nat-useful-tac}

More recurrences, witness from existential, \C{ifP}, comparisons, \C{ring}
\section{See also}




\newpage
