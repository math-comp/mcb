% vim:set tw=70:
% vim:set spell:
% vim:set errorformat="":
\chapter[Mixing data and proofs]{Mixing data and proofs\\[2ex]\Large\itshape What is a sub-type?}

one of the peculiarity of \mcbCIC{} is that proofs are a first
class entity.  one can use this fact to model the operation of building
the set of terms of that type that satisfy a given property.
the idea is to build the type of paris x px.
it is well known that in TT building sets is problematic,
proofs are not irrelevant in general.
hott seems to provide answers to this problem in general
turning the problem upside down, defining the class of more
propositions, and then...
in CIC we don't have a notion of mere proposition as powerful
as in hott, but thanks to a famour result of heideberg
we know that the class of decdable eq, hence decidable predicates
... has uip.  This fits well our framework where all proedicate that
are decidable are expressed as booleans, and p x = true is
the usual way to state things.
this also makes the overloading mechanism we have seen in the prev
chapter perform proof search. eqtype for pairs. but also more
interesting

\mcbsection{Types with a decidable equality}

refine eqtype adding the proposition of compatibility with leibnitz
show that == finds proofs

\mcbsection{Types with an enumeration}

here we cheat, saying that the precise hierarchy is in the next
chapter.  also the nesting of records will be fake.

\mcbsection{Finite sets}

once we have finite types we can carve sets into them and
discuss their main property.

\mcbsection{subtype (proof search)}


Boolean sigma types, records, coercions, UIP. Examples: ordinals,
tuples (not their use which requires CS). This chapter might come
after some presentation of basic canonical structures, i.e. reorganize
the content between this chapter and the next.

example of tuples here? better ordinals?

GG: makes many points here (not fully understood by Enrico):
\begin{itemize}
\item ordinals/tuples are easy to use but hard to build
\item it is a tradeoff, but is not clear if we can give hints on when
	a specific datatype like ordinals is better that unpackaged
	stuff.
\end{itemize}

UIP is advanced, the basic user should just be told that putting
bools inside a record is just fine.

\begin{itemize}
	\item record (flat)
	
\end{itemize}


