\newgeometry{left=1cm,right=1cm,top=1cm,bottom=1.5cm}
\pagestyle{empty}
\begin{landscape}
\begin{small}
\begin{multicols*}{3}[\begin{center}\section*{\underbar{Cheat Sheet}}\end{center}]

\subsection*{Terminology}

$$
\begin{array}{rl}
	Context &
\left\{\begin{minipage}{0.14\textwidth}
\begin{lstlisting}
x : T
S : {set T}
xS : x \in S
\end{lstlisting}
\end{minipage}
\right.
\\
The~bar & \mathrm{~~\C{===============}}
\\
Goal &
\left\{
\begin{minipage}{0.20\textwidth}
\begin{lstlisting}
forall y, y == x -> y \in S
\end{lstlisting}
\end{minipage}
\right.
\\
& \raisebox{0.8em}[0em][0em]{$
~~\underbrace{~~~~~~~~~~~~~~~~}_{}~~~~\underbrace{~~~~~~~~~~~~}_{}$}
\\
& ~~Assumptions~~~~~Conclusion
\end{array}
$$

\begin{description}
\item[Top] is the first assumption, \C{y} here
\item[Stack] is an alternative name for the list of $Assumptions$
\end{description}

\subsection*{Popping from the stack}
Note: in the following we assume $cmd$ does nothing, exactly like
\C{move}.

\begin{cheat}
$cmd$=> x px
\end{cheat}
Run $cmd$, then pop Top, put it in the context, naming it \C{x}, then
pop the new Top and name it \C{px} in the context

\begin{cheatout}
=========
 forall x,
  P x -> Q x -> G
$~$
\end{cheatout}
$\to$
\begin{cheatout}
 x : T
 px : P x
=========
 Q x -> G
\end{cheatout}

\begin{cheat}
$cmd$=> [|x xs] //
\end{cheat}
  Run $cmd$, then reason by cases on Top. In the first
  branch do nothing, in the second one pop two
  assumptions naming them \C{x} and \C{xs}. Then get rid of
  trivial goals. Note that, since only the first branch is
  trivial, one can write \C{=> [// | x xs]} too.\caveat{
  Immediately after
  \C{case} and \C{elim} it does not perform any case analysis, but
  can still introduce different names in different branches}

\begin{cheatout}
=========
 forall s : seq nat,
  0 < size s -> P s
$~$
\end{cheatout}
$\to$
\begin{cheatout}
 x : nat
 xs : seq nat
=========
 0 < size (x :: xs) -> P (x :: xs)
\end{cheatout}

\begin{cheat}
$cmd$=> /andP[pa pb]
\end{cheat}
  Run $cmd$, then apply the view \C{andP} to Top, then destruct the
  conjunction and introduce the two
  parts into the context, naming them \C{pa} and \C{pb}

\begin{cheatout}
=========
 A && B -> C -> D
$~$
$~$
\end{cheatout}
$\to$
\begin{cheatout}
 pa : A
 pb : B
=========
 C -> D
\end{cheatout}

\begin{cheat}
$cmd$=> /= {px}
\end{cheat}
  Run $cmd$, then simplify the goal, then discard \C{px} from
  the context

\begin{cheatout}
 x : nat
 px : P x
=========
 true && Q x -> R x
\end{cheatout}
$\to$
\begin{cheatout}
 x : nat
=========
 Q x -> R x
$~$
\end{cheatout}

\begin{cheat}
$cmd$=> [y -> {x}]
\end{cheat}
  Run $cmd$, then destruct the existential, introduce y,
  then rewrite with Top from left to right,
  discard the equation, and clear x

\begin{cheatout}
 x : nat
=========
 (exists2 y,
   x = y & Q y) -> P x
\end{cheatout}
$\to$
\begin{cheatout}
 y : nat
=========
 Q y -> P y
$~$
\end{cheatout}

\begin{cheat}
$cmd$=> /(_ x)$\;$ h
\end{cheat}
Introduce \C{h} specialized to \C{x}


\begin{cheatout}
 P : nat -> Prop
 x : nat
======================
 (forall n, P n) -> G
$~$
\end{cheatout}
$\to$
\begin{cheatout}
 P : nat -> Prop
 x : nat
 h : P x
=================
 G
\end{cheatout}


\subsection*{Pushing to the stack}
Note: in the following $cmd$ is not
\C{apply} or \C{exact}. Moreover we display the
goal just before $cmd$ is run.

\begin{cheat}
$cmd$: (x)$\;\;$y
\end{cheat}
  Push \C{y} then push \C{x} on the stack. \C{y} is also cleared

\begin{cheatout}
 x, y : nat
 px : P x
=========
 Q x y
\end{cheatout}
$\to$
\begin{cheatout}
 x : nat
 px : P x
=========
 forall x0 y, Q x0 y
\end{cheatout}


\begin{cheat}
$cmd$: {-2}x (erefl x)
\end{cheat}
  Push the type of \C{(erefl x)}, then push \C{x} on the stack
  binding all but the second occurrence

\begin{cheatout}
 x : nat
=========
 P x
$~$
\end{cheatout}
$\to$
\begin{cheatout}
 x : nat
=========
 forall x0,
  x0 = x -> P x0
\end{cheatout}


\columnbreak

\begin{cheat}
$cmd$: _.+1 {px}
\end{cheat}
  Clear \C{px} and generalize the goal with
  respect to the first match of the pattern \C{_.+1}

\begin{cheatout}
 x : nat
 px : P x
=========
 x < x.+1
\end{cheatout}
$\to$
\begin{cheatout}
 x : nat
=========
 forall x0, x < x0
$~$
\end{cheatout}


\subsection*{Proof commands}

\begin{cheat}
rewrite Eab (Exc b).
\end{cheat}
  Rewrite with \C{Eab} left to right, then with \C{Exc} by
  instantiating the first argument with \C{b}

\begin{cheatout}
 Eab : a = b
 Exc : forall x, x = c
=========
 P a
\end{cheatout}
$\to$
\begin{cheatout}
 Eab : a = b
 Exc : forall x, x = c
=========
 P c
\end{cheatout}

\begin{cheat}
rewrite -Eab {}Eac.
\end{cheat}
Rewrite with \C{Eab} right to left then with \C{Eac}
left to right, finally clear \C{Eac}

\begin{cheatout}
 Eab : a = b
 Eac : a = c
=========
 P b
\end{cheatout}
$\to$
\begin{cheatout}
 Eab : a = b
=========
 P c
$~$
\end{cheatout}

\columnbreak

\begin{cheat}
rewrite /(_ && _).
\end{cheat}
  Unfold the definition of \C{&&}

\begin{cheatout}
 a : bool
=========
 a && a = a
$~$
\end{cheatout}
$\to$
\begin{cheatout}
 a : bool
=========
 if a then a
   else false = a
\end{cheatout}

\begin{cheat}
rewrite /= -[a]/(0+a)$\;$ -/c.
\end{cheat}
  Simplify the goal, then change \C{a} into \C{0+a}, finally fold back
  the local definition \C{c}

\begin{cheatout}
 a, b : nat
 c := b + 3 : nat
=========
true && (a == b + 3)
\end{cheatout}
$\to$
\begin{cheatout}
 a, b : nat
 c := b + 3 : nat
=========
0 + a == c
\end{cheatout}

\begin{cheat}
apply: H.
\end{cheat}
  Apply \C{H} to the current goal

\begin{cheatout}
 H : A -> B
=========
 B
\end{cheatout}
$\to$
\begin{cheatout}
=========
 A
$~$
\end{cheatout}

\begin{cheat}
case: ab.
\end{cheat}
  Eliminate the conjunction or disjunction

\begin{cheatout}
 ab : A /\ B
=========
 G
\end{cheatout}
$\to$
\begin{cheatout}
=========
 A -> B -> G
$~$
\end{cheatout}

\begin{cheat}
case: (leqP a b).
\end{cheat}
Reason by cases using the \C{leqP} spec lemma

\begin{cheatout}
 a, b : nat
=========
 a <= b && b > a
\end{cheatout}
$\to$
\begin{cheatout}
 a, b : nat
=========
 true && false
\end{cheatout}

\noindent\hspace{0.24\textwidth}
\begin{cheatout}
 a, b : nat
=========
 false && true
\end{cheatout}


\begin{cheat}
elim: s.
\end{cheat}
Perform an induction on {\tt s}


\begin{cheatout}
 s : seq nat
=========
 P s
\end{cheatout}
$\to$
\begin{cheatout}
=========
 P [::]
\end{cheatout}

\noindent\hspace{0.24\textwidth}
\begin{cheatout}
=========
 forall n s,
  P s -> P (n :: s)
\end{cheatout}

\begin{cheat}
elim/last_ind: s
\end{cheat}
  Start an induction on \C{s} using the induction
  principle \C{last_ind}

\begin{cheatout}
 s : seq nat
=========
 P s
\end{cheatout}
$\to$
\begin{cheatout}
=========
 P [::]
\end{cheatout}

\noindent\hspace{0.24\textwidth}
\begin{cheatout}
=========
 forall s x,
  P s ->
   P (rcons s x)
\end{cheatout}

\columnbreak

\begin{cheat}
have pa : P a.
\end{cheat}
  Open a new goal for \C{P a}. Once it is resolved,
  introduce a new entry for it in the context, named \C{pa}

\begin{cheatout}
 a : T
=========
 G
\end{cheatout}
$\to$
\begin{cheatout}
 a : T
=========
 P a
\end{cheatout}

\noindent\hspace{0.24\textwidth}
\begin{cheatout}
 a : T
 pa : P a
=========
 G
\end{cheatout}

\begin{cheat}
by [].
\end{cheat}
  Prove the goal by trivial means, or fail

\begin{cheatout}
=========
 0 <= n
\end{cheatout}
$\to$
\begin{cheatout}
\end{cheatout}

\begin{cheat}
exact: H.
\end{cheat}
  Apply \C{H} to the current goal and then assert all
  remaining goals, if any, are trivial. Equivalent to
  \C{by apply: H.}

\begin{cheatout}
 H : B
=========
 B
\end{cheatout}
$\to$
\begin{cheatout}
\end{cheatout}


%\columnbreak

\subsection*{Reflect and views}

\begin{cheat}
reflect P b
\end{cheat}
States that \C{P} is logically equivalent to \C{b}.

\columnbreak

\begin{cheat}
apply: (iffP V)
\end{cheat}

Proves a reflection goal, applying the view
  lemma \C{V} to the propositional part of \C{reflect}.
  E.g. \C{apply: (iffP idP)}


\begin{cheatout}
 P : Prop
 b : bool
=============
 reflect P b
\end{cheatout}
$\to$
\begin{cheatout}
 P : Prop
 b : bool
=========
 b -> P
\end{cheatout}

\noindent\hspace{0.24\textwidth}
\begin{cheatout}
 P : Prop
 b : bool
=========
 P -> b
\end{cheatout}


\begin{cheat}
apply/V1/V2
\end{cheat}
Prove boolean equalities by considering them as
  logical double implications. The term \C{V1} (resp. \C{V2}) is the
  view lemma applied to the left (resp. right) hand
  side. E.g. \C{apply/idP/negP}


\begin{cheatout}
 b1 : bool
 b2 : bool
=========
 b1 = ~~ b2
\end{cheatout}
$\to$
\begin{cheatout}
 b1 : bool
 b2 : bool
=========
 b1 -> ~ b2
\end{cheatout}

\noindent\hspace{0.24\textwidth}
\begin{cheatout}
 b1 : bool
 b2 : bool
=========
 ~ b2 -> b1
\end{cheatout}


\begin{cheat}
rewrite: (eqP Eab)
\end{cheat}
rewrite with the boolean equality \C{Eab}


\begin{cheatout}
 Eab : a == b
=============
 P a
\end{cheatout}
$\to$
\begin{cheatout}
 Eab : a == b
=========
 P b
\end{cheatout}


\subsection*{Idioms}

\begin{cheat}
case: b => [h1| h2 h3]
\end{cheat}
  Push \C{b}, reason by cases, then pop
  \C{h1} in the first branch and \C{h2} and \C{h3} in the second

\begin{cheat}
have /andP[x /eqP->] : P a && b == c
\end{cheat}
  Open a subgoal for \C{P a && b == c}. When proved,
  apply to it the \C{andP} view, destruct the conjunction,
  introduce \C{x}, apply the view \C{eqP} to turn
  \C{b == c} into \C{b = c}, then rewrite with it and discard the
  equation

\begin{cheat}
elim: n.+1 {-2}n (ltnSn n) => {n}// n
\end{cheat}
  General induction over \C{n}, note that the first goal
  has a false assumption \C{forall n, n < 0 -> ...} and is
  thus solved by \C{//}

\begin{cheatout}
 n : nat
=========
 P n
\end{cheatout}
$\to$
\begin{cheatout}
 n : nat
=========
(forall m, m < n -> P m) ->
  forall m, m < n.+1 -> P m
\end{cheatout}


\begin{cheat}
rewrite lem1 ?lem2 //
\end{cheat}
  Use the equation with premises \C{lem1}, then
  get rid of the side conditions with \C{lem2}

\subsection*{Searching}

\begin{cheat}
Search _ addn (_ * _)$\;$ "C"$\;$ in ssrnat
\end{cheat}
  Search for all theorems with no constraints on
  the main conclusion (conclusion head symbol is the wildcard \C{_}),
  that talk about the \C{addn} constant, matching anywhere the pattern
  \C{(_ * _)} and having a name containing the string \C{"C"} in the
  module \C{ssrnat}

% \subsection*{Misc notations}
% \begin{lstlisting}
% "f1 \o f2" := (comp f1 f2)
% "x \in A" := (in_mem x (mem A))
% "x \notin A" := (~~ (x \in A))
% "[ /\ P1 , P2 & P3 ]" := (and3 P1 P2 P3)
% "[ \/ P1 , P2 | P3 ]" := (or3 P1 P2 P3)
% "[ && b1 , b2 , .. , bn & c ]" :=
%   (b1 && (b2 && .. (bn && c) .. ))
% "[ || b1 , b2 , .. , bn | c ]" :=
%   (b1 || (b2 || .. (bn || c) .. ))
% "#| A |" := (card (mem A))
% "n .-tuple" := (tuple_of n)
% "'I_ n" := (ordinal n)
% "f1 =1 f2" := (eqfun f1 f2)
% "b1 (+) b2" := (addb b1 b2)
% \end{lstlisting}
% 
% \subsection*{Notations for natural numbers: \C{nat}}
% \begin{lstlisting}
% "n .+1" := (succn n)
% "n .-1" := (predn n)
% "m + n" := (addn m n)
% "m - n" := (subn m n)
% "m <= n" := (leq m n)
% "m < n"  := (m.+1 <= n)
% "m <= n <= p" := ((m <= n) && (n <= p))
% "m * n" := (muln m n)
% "n .*2" := (double n)
% "m ^ n" := (expn m n)
% "n `!" := (factorial n)
% "m %/ d" := (divn m d)
% "m %% d" := (modn m d)
% "m == n %[mod d ]" := (m %% d == n %% d)
% "m %| d" := (dvdn m d)
% "pi .-nat" := (pnat pi)
% \end{lstlisting}
% 
% \subsection*{Notations for lists: \C{seq T}}
% \begin{lstlisting}
% "x :: s" := (cons _ x s)
% "[ :: ]" := nil
% "[ :: x1 ]" := (x1 :: [::])
% "[ :: x & s ]" := (x :: s)
% "[ :: x1 , x2 , .. , xn & s ]" :=
%   (x1 :: x2 :: .. (xn :: s) ..)
% "[ :: x1 ; x2 ; .. ; xn ]" :=
%   (x1 :: x2 :: .. [:: xn] ..)
% "s1 ++ s2" := (cat s1 s2)
% \end{lstlisting}
% 
% 
% \subsection*{Notations for iterated operations}
% \begin{lstlisting}
% "\big [ op / idx ]_ i F" :=
% "\big [ op / idx ]_ ( i | P ) F" :=
% "\big [ op / idx ]_ ( i <- r | P ) F" :=
% "\big [ op / idx ]_ ( m <= i < n | P ) F" :=
% "\big [ op / idx ]_ ( i < n | P ) F" :=
% "\big [ op / idx ]_ ( i \in A | P ) F" :=
% "\sum_ i F" :=
% "\prod_ i F" :=
% "\max_ i F" :=
% "\bigcap_ i F" :=
% "\bigcup_ i F" :=
% \end{lstlisting}

\caveat{in the general form, the iterated operation \C{op}
is displayed in prefix form (not in infix form)}
\caveat{the string \C{"big"} occurs in every lemma concerning
iterated operations}

\subsection*{Rewrite patterns}

\begin{cheat}
rewrite [pat]lem [in pat2]lem2 [X in pat3]lem3
\end{cheat}
  Rewrite the subterms selected by the pattern \C{pat} with \C{lem}.
  Then in the subterms selected by the pattern \C{pat2}
  match the pattern inferred from the left-hand side of \C{lem2} and
  rewrite the terms selected by it. Lastly, in the subterms selected by
  \C{pat3} rewrite with \C{lem3} the subterms identified by \C{X} exactly

\begin{cheat}
rewrite {3}[in X in pat1]lem1
\end{cheat}
  Like in \C{rewrite [X in pat1]lem1} but use the pattern
  inferred from \C{lem1} to identify the subterms of \C{X} to be
  rewritten. Of these terms, rewrite only the third one. Example:
  \C{rewrite {3}[in X in f _ X]E.}

\begin{cheatout}
E : a = c
=========
a + f a (a + a) =
  f a (a + a) + a
\end{cheatout}
$\to$
\begin{cheatout}
E : a = c
=========
a + f a (a + a) =
  f a (c + a) + a
\end{cheatout}



\begin{cheat}
rewrite [e in X in pat1]lem1
\end{cheat}
  Like before, but override the pattern inferred from \C{lem1} with
  \C{e}

\begin{cheat}
rewrite [e as X in pat1]lem1
\end{cheat}
  Like \C{rewrite [X in pat1]lem1} but match
  \C{pat1}[\C{X} := \C{e}] instead of just \C{pat1}

\begin{cheat}
rewrite /def1 -[pat]/term /=
\end{cheat}
  Unfold all occurrences of \C{def1}. Then match the goal against \C{pat}
  and change all its occurrences into \C{term} (pure computation). Last,
  simplify the goal.

\begin{cheat}
rewrite 3?lem2 // {hyp} => x px
\end{cheat}
  Rewrite from 0 to 3 times with lem2, then
  try to solve with \C{by []} all the goals. Finally clear \C{hyp}
  and introduce \C{x} and \C{px}

\subsection*{Pattern matching detailed rules}

\begin{description}
\item[pattern] a term, possibly containing \C{_}
\item[key] The head symbol of a pattern
\end{description}

The subterms selected by a pattern:
\begin{enumerate}
\item the goal is traversed outside in, left to right, looking for
      verbatim occurrences of the key
\item the subterms whose key matches verbatim are higher order
      matched (i.e. up to definition unfolding and recursive function
      computation) against the pattern
\item if the matching fails, the next subterm whose key matches is tried
\item if the matching succeeds, the subterm is considered to be the
      (only) instance of the pattern
\item the subterms selected by the pattern are then all the copies of the
      instance of the pattern
\item these copies are searched looking again at the key, and higher order
      comparing the arguments pairwise
\end{enumerate}

Note: occurrence numbers can be combined with patterns. They refer
to the list of of subterms selected by the (last) pattern (i.e. they
are processed at the very end).

\begin{cheat}
set n := {2 4}(_ + b)
\end{cheat}
  Put in the context a local definition named \C{n} for the
  second and fourth occurrences of the subterms selected by the
  pattern \C{(_ + b)}

\begin{cheatout}
=========
(a + b) + (a + b) =
 (a + b) + (0 + a + b)
\end{cheatout}
$\to$
\begin{cheatout}
 n := a + b
=========
(a + b) + n =
  (a + b) + n
\end{cheatout}


\end{multicols*}

\end{small}
\end{landscape}
\restoregeometry
%\pagestyle{headings}
