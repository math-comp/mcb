% Example of Macros to be defined in order to have rt work

% Define snippets (hence prints nothing in the tex)
\newcommand{\rtdef}[2]{}   % #1, #2 are reserved

% Calls Coq on a sequence of snippets (hence prints nothing in the tex)
\newcommand{\rtrun}[2]{}   % #1, #2 are reserved

% Lets one declare the expected output #3, rt checks it is almost-equal
% to something produced by a run. #2 are extra args for the macro (not used in
% the example)
\newcommand{\rtout}[3]{#3} % #1 is reserved, #2 and #3 to be used

% Lets one present snippets; rt asserts such snippets is almost-equal to
% the original ones.  One can also define the snippet on the fly
\newcommand{\rtprt}[3]{#3} % #1 is reserved, #2 and #3 to be used

% As above, but environments
\newbox{\devnull}
\newenvironment{rtdefenv}[1]{\savebox{\devnull}\bgroup}{\egroup}
\newenvironment{rtprtenv}[2]{}{}
\newenvironment{rtoutenv}[2]{}{}
